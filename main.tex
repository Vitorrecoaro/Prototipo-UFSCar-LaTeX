\documentclass[12pt]{article}
\usepackage[utf8]{inputenc}
\usepackage[portuguese]{babel}
\usepackage[T1]{fontenc}
\usepackage{float}
\usepackage{graphicx}
\usepackage{enumitem}

% Para usar os símbolos dos conjuntos.
\usepackage{amssymb}

% Para fazer equações alinhadas.
\usepackage{amsmath}

% Renomeando o comando \R.
\newcommand{\R}{\mathbb{R}}

% Renomeando o comando \C.
\newcommand{\C}{\mathbb{C}}

% Define o caminho para as imagens.
\graphicspath{Imagens/}

% Identação da primeira linha do parágrafo.
\usepackage{indentfirst}

% Criar links no documento.
\usepackage{hyperref}

% Definindo layout do documento.
    
    % Margens
    \usepackage{geometry}
    \geometry{top = 3cm, left = 3cm, right = 2cm, bottom = 2cm}
    
    % Espaçamento.
    \renewcommand{\baselinestretch}{1.5}
\begin{document}

% Inclui a capa no documento.
% Capa.

% Remove o número da página.
\thispagestyle{empty}

\begin{center}

    \textbf{UNIVERSIDADE FEDERAL DE SÃO CARLOS}

    \vspace*{5mm}

    Alunos:
    

    \vspace*{\fill}\textbf{TÍTULO - Subtítulo}

    \vspace{\fill}\textbf{São Carlos}

    \textbf{Ano de publicação}

\end{center}

% Cria uma nova página.
\newpage

% Começa a contagem das páginas a partir daqui.
\setcounter{page}{1}

% Parte principal do documento.
\section{Introdução}

\end{document}
